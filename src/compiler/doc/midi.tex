\documentclass[11pt,a4paper]{article}
\usepackage{graphicx}
\usepackage{colortbl}
\usepackage{fullpage}
\usepackage{parskip}
\usepackage{booktabs}
\usepackage{moreverb}
\usepackage{nicefrac}

\newenvironment{expose}{\vskip3mm\qquad\begin{raggedright}}{%
\end{raggedright}\vskip3mm}

\input floats.inc

\begin{document}

\title{Using MIDI in Flickernoise Patches}
\author{Werner Almesberger \\
  {\normalsize werner@almesberger.net}}

\maketitle

%\setcounter{tocdepth}{1}
%\tableofcontents

% -----------------------------------------------------------------------------


\section{Introduction}

This document describes the mechanisms Flickernoise provides for
interacting with MIDI controls and explains how patches can make use of
this functionality.

Note that this work is still in progress. For example,
\begin{itemize}
  \item the MIDI device specifications should not have to be part of
    patches,
  \item Flickernoise should only consider devices that are really present, and
  \item we should support other events than just MIDI control message,
    such as other MIDI message types, DMX, keyboard, IR remote, etc.
\end{itemize}


% -----------------------------------------------------------------------------


\section{Quick start}

While the finer details of MIDI controls can get complicated, the
items in the following example are often all that is needed to
use many MIDI devices:

\begin{listing}{1}
midi "Gizmo" {
	main = fader(102);
	aux = pot(103);
	but1 = button(16);
	but2 = button(17);
	select = switch(24);
}

sensitivity = range(main);

per_frame:
	wave_scale = sensitivity*10;
	sensitivity = sensitivity*0.99;
\end{listing}

In lines 1 through 6 we describe the controls the MIDI device called
``Gizmo'' provides. In this case, we have one fader, one potentiometer,
two buttons, and one switch. We assign them names that represent their
role: {\tt main}
for the principal fader, {\tt aux} for the potentiometer, and so on.

The numbers are the respective MIDI controller numbers the device uses.
They can be found with the MIDI monitor function of the ``MIDI settings''
dialog in Flickernoise.

In line 9, we bind the {\tt main} control to a variable. This variable
receives the value 0 if the fader is at its minimum and 1 at its maximum.
It is
then used in per-frame and per-vertex equations. As line 13 shows, one
can also change this variable, e.g., to make the sensitivity slowly decay
if there is no input from the MIDI device.


% -----------------------------------------------------------------------------


\figarch


\section{Architecture}

Figure \ref{arch} shows how MIDI messages are processed in Flickernoise:
For each MIDI device, an entry in the device database is selected (1).
This entry describes the characteristics of the elements of the MIDI
device, e.g., what kind of control elements they are and what
addresses they use.

A patch using MIDI devices binds control elements to variables (2).
When binding, the patch specifies how it expects the element to
behave, e.g., whether it should produce continuous values between
0 and 1 or just 0 when ''off'' and 1 when ''on''.

Event messages from the MIDI device (3) identify the element they
belong to and carry a numeric value indicating the element's state.
This value is translated (4) according to the element's characteristics
from the device database combined with the expected function from
the patch.

Finally, the result is stored in the control variable (5) where it can
then be used by the equations in the patch.


% - - - - - - - - - - - - - - - - - - - - - - - - - - - - - - - - - - - - - - -


\subsection{MIDI control messages}

A MIDI device sends a message each time one of its elements is actuated.
There are various types of elements, keys, pitch wheels, controls, etc.
Controls can be all sorts of things, including faders, potentiometers,
and push buttons.

For now, we only consider controls. When a control is actuated, the
MIDI device generates a MIDI Control Change message of the structure
shown in figure \ref{msg}.

Channel numbers are encoded as values from 0 to 15 in the actual
MIDI message but are commonly presented as 1 to 16 to the user.
Also Flickernoise follows this convention.

Most MIDI devices use a single channel. The number of this channel
can sometimes be set by the user. The controller numbers are typically
fixed.

When Flickernoise receives a MIDI message, it
uses the message type, the channel number, and the controller
number to determine where to send the value. 

\figmsg


% - - - - - - - - - - - - - - - - - - - - - - - - - - - - - - - - - - - - - - -


\subsection{Control variables}

Patches communicate with the outside world through variables. MIDI
control input is no exception. Instead of using pre-defined names
like it is the case for {\tt time}, {\tt bass}, etc., the names of
MIDI control variables can be chosen freely.

Values from MIDI controls are usually translated to
the range 0 to 1. This can be as simple a division by 127.
Section \ref{binding} describes more ways to translate MIDI messages.

Updates of control variables are synchronized with patch execution
such that updates never happen while the patch is running.


% -----------------------------------------------------------------------------


\section{Using controls in patches}

The following sections describe the syntax and semantics of the
language constructs that give access to MIDI controls.


% - - - - - - - - - - - - - - - - - - - - - - - - - - - - - - - - - - - - - - -


\subsection{Device database}

The device database tells Flickernoise how to identify MIDI
devices and their elements and how the elements behave. It also
assigns names to the elements that are then used to refer to
them in patches.

Device specifications are added to the device database with 
a MIDI device entry that looks as follows:

\begin{expose}
{\tt midi} {\em identification} \verb"{" {\em element $\ldots$} \verb"}"
\end{expose}

Each element in the device entry has the following syntax:

\begin{expose}
{\tt {\em name} =
  {\em device\_type}($\left[\hbox{\em channel}\verb","\right]$
  {\em control\_number});}
\end{expose}

{\em device\_type} provides a broad characterization of the control
element and can be {\tt fader}, {\tt pot}, {\tt differential},
{\tt button}, or {\tt switch}. Control elements are discussed in
detail in section \ref{controls}.

{\em channel} is the MIDI channel number, from 1 to 16. If the channel
is omitted, the element will match any channel.

{\em control\_number} is a number from 0 to 127 the MIDI device uses
to identify the control element.

This is the example from the quick start section with channel numbers
added:

\begin{listing}{1}
midi "Gizmo" {
	main = fader(1, 102);
	aux = pot(1, 103);
	but1 = button(1, 16);
	but2 = button(1, 17);
	select = switch(1, 24);
}
\end{listing}


% - - - - - - - - - - - - - - - - - - - - - - - - - - - - - - - - - - - - - - -


\figbind

\subsection{Binding}
\label{binding}

In order to use a control in a patch, we have to establish a connection
between the control element (in the device database) and a patch
variable. We call such a variable a {\em control variable}.

Control variables are bound with a construct that looks like a
variable assignment:

\begin{expose}
{\tt {\em control\_variable} = {\em function}({\em element\_name});}
\end{expose}

{\em control\_variable} can be a pre-defined per-frame or per-vertex
variable or it can be a user-defined variable. Variables that are
updated by Flickernoise itself cannot be used as control variables.
These variables are
\input nocvars.inc

{\em function} describes how the patch expects the control to behave.
Flickernoise then tries to adapt the behaviour of the actual device
to what the patch expects. The following functions are available:

\begin{description}
  \item[\tt range]
    The control variable has a value between 0 and 1, depending on the
    setting of the device. This is commonly used for faders and
    potentiometers.
  \item[{\tt unbounded}, {\tt cyclic}]
    These are special functions used with some rotary encoders. They
    are described in detail in section \ref{diff}. With other control
    elements, they behave just like {\tt range}.
  \item[\tt button]
    The control variable receives the value 1 when the button is
    pressed and returns to zero when it is released. When applied to
    elements that send values between 0 and 1, the value is rounded.
  \item[\tt switch]
    The control variable can be set to 0 or 1 and retains the value
    until the element is actuated again.
\end{description}

{\em element\_name} is the name of the control element, as in the
device database.

The result of binding is a stimulus entry that tells Flickernoise
how to interpret incoming messages.
Figure \ref{bind} illustrates the steps in creating the stimulus
entry: Flickernoise
selects a control element with the desired name from the available
devices (1). The information needed to identify messages from this
control element is copied to the stimulus entry (2). Using the
device type from the element record (3) and the function from the
binding instruction, a suitable translator is selected from the
translation map. Also this is recorded in the stimulus entry (4).
Finally, the variable is looked up or added to the patch (5) and
a reference to it is placed in the stimulus entry.


% - - - - - - - - - - - - - - - - - - - - - - - - - - - - - - - - - - - - - - -


\figstim

\subsection{Event processing}

Figure \ref{stim} illustrates how events are translated into changes
of control variables.

Each time a MIDI Control Change message arrives, Flickernoise looks for a
stimulus that matches the message type, i.e., MIDI control, the channel
number and the controller number (1). If no matching entry exists,
the message is ignored.

The stimulus tells Flickernoise which translator to use and where
to store the result. In our example, we have a {\tt pot} to {\tt range}
translation (2), which is simply a division by 127. The result,
$\nicefrac{28}{127}=0.220$, is stored in the location of the variable
{\tt var} (3).


% - - - - - - - - - - - - - - - - - - - - - - - - - - - - - - - - - - - - - - -


\subsection{Writing to control variables}

A patch can overwrite the values of control variables. In the case of
per-frame variables, they retain their modified content across
frames until a new event arrives. The new event will always overwrite
the previous value of the variable without regard for any changes that
may have been made.

Changes to per-vertex variables are lost at the end of the program.

It is sometimes desirable to make a control change a variable relative
to a value set by the patch. The following example illustrates how
this can be accomplished:

\begin{listing}{1}
last = 0;
cvar = range(foo);

per_frame:
	var = var+cvar-last;
	last = cvar;
	/* ... use "var" ... */
	var = condition ? new_value : var;
\end{listing}

Instead of writing to the control variable {\tt cvar} directly, we
propagate any relative changes of {\tt cvar} to {\tt var} in lines
5 and 6, and only modify {\tt var} in line 8.

Note that {\tt var} may assume values outside the range $0\ldots 1$.
If this is not desirable, it can be clipped after line 5 with

\begin{listing}{1}
	var = max(min(var, 1), 0);
\end{listing}


% - - - - - - - - - - - - - - - - - - - - - - - - - - - - - - - - - - - - - - -


\subsection{Multiple bindings}

So far, we have only seen one element being bound to one control
variable at a time. A control variable can also be bound to
multiple elements and the same element can be bound to multiple
control variables.

Binding the same element to different variables can be useful in
cases where the translation differs as well. For example, the
following code snippet would control two parameters with the same
element:

\begin{listing}{1}
growth = range(main);
tilt = cyclic(main);

per_vertex:
	zoom = growth*0.2;
	angle = 2*3.14159*tilt;
	wave_x = cx+0.1*cos(angle);
	wave_y = cy-0.1*sin(angle);
\end{listing}

A situation that may be even more common is to have multiple elements
that change the same variable, e.g., when using multiple input
devices. Example:

\begin{listing}{1}
midi "foo" {
        foo_pot = pot(12);
}

midi "bar" {
        bar_pot = pot(34);
}

sensitivity = range(foo_pot);
sensitivity = range(bar_pot);

per_frame:
	wave_scale = sensitivity*20;
\end{listing}


% -----------------------------------------------------------------------------


\section{Control elements}
\label{controls}

In the sections below, we describe the various control elements, how
they are described in the device database, and their behaviour. We
give examples that show the physical state of the element, the value
a MIDI device may typically send for the element in that state, and
then the resulting values for the various translations.

Since {\tt range}, {\tt unbounded}, and {\tt cycle} only differ from
each other in one case, they are usually abbreviated to just
``{\tt range}, $\ldots$''.


% - - - - - - - - - - - - - - - - - - - - - - - - - - - - - - - - - - - - - - -


\subsection{Faders}

Faders are slide potentiometers that normally cover the whole
range of values a MIDI controller can send: 0--127. They
retain their position when released.

Fader elements of a device are declared with {\tt fader()} and
are typically bound with {\tt range()}:

\begin{listing}{1}
midi ... {
	name = fader(...);
}

var = range(name);
\end{listing}

The following example shows how Flickernoise maps faders to control
variables. We start with the fader in the 0\% position, move it to
the 40\%, then 60\%, and the 100\% position. At the end, we return
it to the 0\% position.

\begin{expose}
\begin{tabular}{lcccccl}
  \raisebox{6mm}{User input} &
  \includegraphics{fader-0.pdf} &
  \includegraphics{fader-40.pdf} &
  \includegraphics{fader-60.pdf} &
  \includegraphics{fader-100.pdf} &
  \includegraphics{fader-0.pdf} \\
  \cmidrule(r){1-6}
  MIDI value &
  & 51 & 76 & 127 & 0 \\
  \midrule
  Translation
  & 0 & 0.4 & 0.6 & 1 & 0 & \tt range, $\ldots$ \\
  & 0 & 0   & 1   & 1 & 0 & \tt button, switch \\
\end{tabular}
\end{expose}

The mapping is quite straightforward: {\tt range}, {\tt unbounded},
and {\tt cycle} produce a value from 0 to 1 corresponding to the
position of the knob. {\tt button} and {\tt switch} produce 0 if
the knob is in the lower half of the range, 1 if it is in the upper half.


% - - - - - - - - - - - - - - - - - - - - - - - - - - - - - - - - - - - - - - -


\subsection{Rotary potentiometers}

Rotary potentiometers work exactly like faders except that they are
declared with {\tt pot()}:\footnote{The current implementation does
not distinguish at all between {\tt fader} and {\tt pot}, but they
may be represented with different symbols in a future GUI.}

\begin{listing}{1}
midi ... {
	name = pot(...);
}

var = range(name);
\end{listing}

They have mechanical stops at the beginning and at the end of
their range, which distinguishes them from the rotary encoders described
in the next section.

The example below shows a potentiometer that travels over an angle
of $270^\circ$:

\begin{expose}
\begin{tabular}{lccccl}
  \raisebox{6mm}{User input} &
  \includegraphics{pot-0.pdf} &
  \includegraphics{pot-90.pdf} &
  \includegraphics{pot-180.pdf} &
  \includegraphics{pot-270.pdf} \\
  \cmidrule(r){1-5}
  MIDI value &
  & 42 & 85 & 127  \\
  \midrule
  Translation
  & 0 & 0.33 & 0.67 & 1 &  \tt range, $\ldots$ \\
  & 0 & 0    & 1   & 1 &  \tt button, switch \\
\end{tabular}
\end{expose}


% - - - - - - - - - - - - - - - - - - - - - - - - - - - - - - - - - - - - - - -


\subsection{Rotary encoders acting as potentiometers}
\label{encpot}

Rotary encoders look similar to potentiometers but differ from them
by not having a mechanical stop. This means that they can be turned
indefinitely in the same direction.

MIDI devices usually emulate the behaviour of potentiometers by ignoring
any turns at the end of the value range. When the direction is reversed,
the values change immediately.

Rotary encoders acting as potentiometers are also declared with
{\tt pot} and bound with {\tt range}.

The example below shows how a rotary encoder covering the full value range
in one $360^\circ$ turn behaves when it is first turned $450^\circ$ clockwise
and then $45^\circ$ counterclockwise:

\begin{expose}
\begin{tabular}{lcccccl}
  \raisebox{6mm}{User input} &
  \includegraphics{enc-0.pdf} &
  \includegraphics{enc-90.pdf} &
  \includegraphics{enc-270.pdf} &
  \includegraphics{enc-450.pdf} &
  \includegraphics{enc-405.pdf} \\
  \cmidrule(r){1-6}
  MIDI value &
  & 32 & 96 & 127 & 121  \\
  \midrule
  Translation
  & 0 & 0.25 & 0.75 & 1 & 0.875 & \tt range, $\ldots$ \\
  & 0 & 0    & 1    & 1 & 1     & \tt button, switch \\
\end{tabular}
\end{expose}

The encoder stops at MIDI value 127 at the third turn and the
remaining $90^\circ$ are ignored.

Note that the the above is a bit simplified. Rotary encoders commonly
found in MIDI devices only have 20--30 positions per full turn. When
turning them slowly, it therefore takes several full turns to cross
the entire range. To permit quick changes, MIDI controllers usually
change the value in steps larger than one when the encoder is turned
rapidly.


% - - - - - - - - - - - - - - - - - - - - - - - - - - - - - - - - - - - - - - -


\subsection{Push buttons}

Push buttons are activated by pressing them and they return to the
inactive state when released. Buttons can only be either fully on or
fully off, without intermediate values.

Push buttons are declared with {\tt button()} and bound with
{\tt button()} or {\tt switch()}:

\begin{listing}{1}
midi ... {
	name = button(...);
}

var = button(name);
\end{listing}

{\tt switch()} turns the control on when the button
is pressed the first time and then off again when pressed a second
time.

Example:

\begin{expose}
\begin{tabular}{lcccccl}
  \raisebox{5mm}{User input} &
  \includegraphics{button-up.pdf} &
  \includegraphics{button-down.pdf} &
  \includegraphics{button-up.pdf} &
  \includegraphics{button-down.pdf} &
  \includegraphics{button-up.pdf} \\
  \cmidrule(r){1-6}
  MIDI value &
  & 127 & 0 & 127 & 0\\
  \midrule
  Translation
  & 0 & 1 & 0 & 1 & 0 & \tt range, $\ldots$, button \\
  & 0 & 1 & 1 & 0 & 0 & \tt switch \\
\end{tabular}
\end{expose}


% - - - - - - - - - - - - - - - - - - - - - - - - - - - - - - - - - - - - - - -


\subsection{Switches}

A switch is set to either on or off and retains its state until
actuated again. Switches are declared with {\tt switch()} and typically
bound with {\tt switch()} as well:

\begin{listing}{1}
midi ... {
	name = switch(...);
}

var = switch(name);
\end{listing}

Switches behave the same for all functions:

\begin{expose}
\begin{tabular}{lcccl}
  \raisebox{4mm}{User input} &
  \includegraphics{switch-off.pdf} &
  \includegraphics{switch-on.pdf} &
  \includegraphics{switch-off.pdf} \\
  \cmidrule(r){1-4}
  MIDI value &
  & 127 & 0 \\
  \midrule
  Translation
  & 0 & 1 & 0 & all \\
\end{tabular}
\end{expose}


% - - - - - - - - - - - - - - - - - - - - - - - - - - - - - - - - - - - - - - -


\subsection{One-way buttons}

Some devices have buttons that send a MIDI event only when pressed but
not when released. E.g.,

\begin{expose}
\begin{tabular}{lcccccl}
  \raisebox{5mm}{User input} &
  \includegraphics{button-up.pdf} &
  \includegraphics{button-down.pdf} &
  \includegraphics{button-up.pdf} &
  \includegraphics{button-down.pdf} &
  \includegraphics{button-up.pdf} \\
  \cmidrule(r){1-6}
  MIDI value &
  & 127 &  & 127 & \\
\end{tabular}
\end{expose}

Since the control variables in Flickernoise reflect the state of a
control and not events, we cannot directly use such devices. However,
by exploiting the ability to overwrite a control variable, we can
simulate a release as follows:

\begin{listing}{1}
midi ... {
	name = button(...);
}

var = button(name);

per_frame:
	/* use "var" */
	var = 0;
\end{listing}

When the button is pressed, it will set {\tt var} to 1. At the next
frame, this value can be used. At the end of the frame (and before
running per-vertex equations), it is reset to zero. {\tt var} thus
acts like a {\tt button} control variable where the button is always
pressed for exactly one frame duration.


% - - - - - - - - - - - - - - - - - - - - - - - - - - - - - - - - - - - - - - -


\subsection{Differential encoders}
\label{diff}

The rotary encoders of some controllers send differential values
instead of the usual absolute values. Differential values allow
Flickernoise not only to translate them to absolute values, but
also enable more sophisticated translations.

Differential values are encoded as signed 7 bit numbers. This means
that values in the range 0--63 represent an increase by that amount
while values in the range 64--127 represent a decrease by 128 minus
the value. E.g., a value of 127 would mean a decrease by one.

Differential encoders are declared with {\tt differential()}:

\begin{listing}{1}
midi ... {
	name = differential(...);
}
\end{listing}

There are three distinct options for binding them to continuous values:
with {\tt range()}, the
emulate potentiometers with a range from 0 to 1. With {\tt unbounded},
the control variable can also assume values below 0 or above 1. Finally,
{\tt cyclic} limits the control variable to a range from 0 to 1 but
makes it wrap from 1 to 0 when increasing, and from 0 to 1 when
decreasing.

\begin{listing}{1}
var1 = range(name1);
var2 = unbounded(name2);
var3 = cyclic(name3);
\end{listing}

This example illustrates the values that result for the different
translations when turning an encoder that needs one $360^\circ$ turn
for the full value range by a total of $405^\circ$:

\begin{expose}
\begin{tabular}{lccccccl}
  \raisebox{6mm}{User input} &
  \includegraphics{dial-0.pdf} &
  \includegraphics{dial-90.pdf} &
  \includegraphics{dial-180.pdf} &
  \includegraphics{dial-360.pdf} &
  \includegraphics{dial-405.pdf} \\
  \cmidrule(r){1-6}
  MIDI value &
  & 32 & 32 & 63 & 16 \\
  \midrule
  Translation
  & 0 & 0.25 & 0.5 & 1 & 1 & \tt range \\
  & 0 & 0.25 & 0.5 & 1 & 1.125 & \tt unbounded \\
  & 0 & 0.25 & 0.5 & 1 & 0.125 & \tt cyclic \\
\end{tabular}
\end{expose}

Continuing the example above, we turn the controller now by
$90^\circ$ counterclockwise:

\begin{expose}
\begin{tabular}{lccl}
  \raisebox{6mm}{User input} &
  \includegraphics{dial-405-idle.pdf} &
  \includegraphics{dial-315.pdf} \\
  \cmidrule(r){1-3}
  MIDI value &
  & 96 \\
  \midrule
  Translation
  & 1 & 0.75 & \tt range \\
  & 1.125 & 0.875 & \tt unbounded \\
  & 0.125 & 0.875 & \tt cyclic \\
\end{tabular}
\end{expose}

Starting from zero again, we now turn the controller first
$45^\circ$ clockwise and then $450^\circ$ counterclockwise:

\begin{expose}
\begin{tabular}{lcccccl}
  \raisebox{6mm}{User input} &
  \includegraphics{dial-0.pdf} &
  \includegraphics{dial-405.pdf} &
  \includegraphics{dial-315.pdf} &
  \includegraphics{dial-135.pdf} &
  \includegraphics{dial-315f.pdf} \\
  \cmidrule(r){1-6}
  MIDI value &
  & 16 & 96 & 64 & 64 \\
  \midrule
  Translation
  & 0 & 0.125 & 0 & 0 & 0 & \tt range \\
  & 0 & 0.125 & -0.125 & -0.625 & -1.125 & \tt unbounded \\
  & 0 & 0.125 & 0.875 & 0.375 & 0.875 & \tt cyclic \\
\end{tabular}
\end{expose}

When emulation a button or a switch, any clockwise turn will set the
control variable to 1 while any counterclockwise turn will set it to
zero:

\begin{expose}
\begin{tabular}{lccccl}
  \raisebox{6mm}{User input} &
  \includegraphics{dial-0.pdf} &
  \includegraphics{dial-90.pdf} &
  \includegraphics{dial-45.pdf} &
  \includegraphics{dial-315.pdf} \\
  \cmidrule(r){1-5}
  MIDI value &
  & 32 & 112 & 96 \\
  \midrule
  Translation
  & 0 & 1 & 0 & 0 & \tt button, switch \\
\end{tabular}
\end{expose}

Like the rotary encoders emulating potentiometers described in section
\ref{encpot}, the values a device sends for differential encoders may be
increased if turning rapidly. Instead of the very large values used in
the examples, one would see a sequence of smaller values as the encoder
is turned.

\end{document}
